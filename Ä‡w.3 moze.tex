\documentclass[12pt,legalpaper,notitlepage]{article}
\usepackage[MeX]{polski}
\usepackage[utf8]{inputenc}
\usepackage{graphicx}
\usepackage{amsmath} %pakiet matematyczny
\usepackage{amssymb} %pakiet dodatkowych symboli
\date{29.10.2021}
\author{Maciej Rutkowski}
\title{Aktywność}

\begin{document}
\maketitle

$$k_{n+1}= n^2+k^{3n+1}_n-k_{n-2}$$
 
$$f(n)=n^4+4n^2-2|_{n=12}$$
 
$$\frac {n!} {k!(n-k)!}=\binom{n}{k}$$
 
$$\frac {\frac {1} {x}+ \frac {1}{y}} {y-z}$$
 
$$\sqrt {\frac {a} {b} + 3}$$
 
$$\sqrt {x-2}$$
 
$$\frac {3!} {4}$$
 
$$\sum_{i=1}^{10} t_i$$
 
$$\mathfrak{ABCabc}$$
 
$$ a\in \mathbb{R}, \mathbf{W} T,\mathbb{N} $$
 
$$2+\left(\frac{3}{4-x}\right)^2$$
 
$$\left\{\frac {2}{3}-cos^2x+\left[4-\left(\frac{2-3}{3-9x}\right)^3\right]^2\right\}^7$$

$$\overline{abc} \ \underline{efg}
\ \widetilde{AFR}$$

\begin{align*}
2x - 5y + 4z &= 8 \\
3x + 9y &= -12
\end{align*}
\end {document}